%============================================================================
\documentclass[letterpaper, 11pt]{article}
\usepackage{comment} % enables the use of multi-line comments (\ifx \fi) 
\usepackage{fullpage} % changes the margin
\usepackage{graphicx} % allows insertion of images
\usepackage{textcomp}

% tight center environment
\newenvironment{tightcenter}{\setlength\topsep{0pt}\setlength\parskip{0pt}
  \begin{center}
}{%
  \end{center}
}

%Custom Commands
\newcommand{\hl}{\begin{center} \line(1,0){468} \end{center}} % lines
\newcommand{\ctitle}[1]{\begin{center} \huge{\textbf{#1}} \end{center}} % custom title

%============================================================================


\begin{document}

%Header 
\noindent
Internet of Things \hfill  Patti Alberto - 895416\\
Project Report\hfill Sciamanna Filippo - 898515

%Title

\ctitle{Smart Thermostat}


\hl
\section*{Project description and requirements}
The objective of this project was to implement a set of sensors, ideally placed in each room of a house, which manage and monitor the temperature using Contiki and simulating it in Cooja.\\
The requirements are as follows:
\begin{itemize}
	\item turn on/off the air conditioning, the heating and the ventilation unit;
	\item capture, store and visualize the temperature of each room;
	\item send an alert when the average temperature is above/below a certain
threshold.   
\end{itemize}
 

\hl 

\section*{Development}
The development of the project is split in two parts:
\begin{itemize}
	\item Contiki and the simulation in Cooja;
	\item Node-RED.
\end{itemize}

\subsection*{Contiki and Cooja}
Contiki and Cooja objective is to manage the thermostats and their interactions with the user.\\
All the nodes are implemented in the simulation using Sky motes, one of the simplest motes available in Cooja, as the operation required are not too complex and demanding to require more powerful nodes. These nodes are divided in two groups based on their role and the operation performed in the network:
\begin{itemize}
	\item RPL Border router (1 mote);
	\item Sensors (N motes, in our simulation we chose to use 4).
\end{itemize}

The RPL Border router node creates a network with the sensors using the RPL algorithm allowing the relaying of message to the right node and allowing the user (node-RED) to reach and retrieve the information of the simulation.\\

The sensor nodes task is to simulate the change in temperature in the room as specified in the project specification and allowing the access to the data to the user.\\
The simulation of the temperature is simply done by using a timeout (20 seconds) after which the value of the temperature is updated based on the state of the node:
\begin{itemize}
	\item +1\textdegree C if the heating is active;
	\item -1\textdegree C if the air conditioning is active;
	\item x2 multiplier of the current status.
\end{itemize} 
In order to avoid reaching unreasonable values it was decided to keep the temperature in the range [0\textdegree C, 50\textdegree C].\\
The communication with the user is implemented using the CoAP protocol by defining two resources:
\begin{itemize}
	\item heating\_opt (actuators/heating);
	\item temperature (sensors/temp).
\end{itemize}

The first one is related to the heating options of the house. A GET request returns the state of the heating device while a POST request allows to manage which device is turned on or off based on a parameter (opt). The accepted value are:
\begin{enumerate}
\setcounter{enumi}{-1}
	\item Turn on/off the air conditioning if the heating is off;
	\item Turn on/off the heating if the air conditioning is off;
	\item Turn on/off the ventilation;
	\item Turn on the air conditioning and turn off the heating;
	\item Turn off the air conditioning;
	\item Turn on the heating and turn off the air conditioning;
	\item Turn off the heating;
	\item Turn on the ventilation;
	\item Turn off the ventilation.
\end{enumerate}
The options \{0, 1, 2\} are meant to be used on a single node while the options \{3, 4, 5, 6, 7, 8\} are meant to be used on multiple nodes in order to override the status. This distinction was made in order to allow Node-RED to control either the single nodes or all the nodes with one operation.\\
The temperature resources is related to the value of the temperature read by the sensors. It is able to handle GET requests that are answered with the current value and it is also an observable resource that sends the data every 5 seconds.

\subsection*{Node-RED}
Node-RED allows the user to see and interact with the sensor network by connecting to the border router and sending CoAP request. In order to keep everything easier to understand, it was decided to take advantage of the possibility to put everything in multiple flows.
\subsubsection*{Room flows}
For each "room" there is a flow where an OBSERVE request on the temperature resource is sent to the corresponding node in order to start receiving the values. Every value received is  stored in the room specific global variable as it is needed to calculate the average values requested in the project specification. Each room also offers the possibility to  send a POST request in order to change the status of the heating devices. These requests use the \{0, 1, 2\} values for the "opt" parameter.
\subsubsection*{House flow}
The house flow simply evaluate the average temperature in the house by periodically taking the last value received by the sensor of each room. The obtained value is then stored in a global variable. There is also the possibility to send POST requests to all the nodes in order to decide the overall behavior of the heating devices by using the value \{3, 4, 5, 6, 7, 8\} for the "opt" parameter.

\subsubsection*{Settings flow}
The settings flow defines multiple forms that allow the user to set the parameters required for the email alerts: a range of value can be set for each room and the house so that an email is sent to the set address when the corresponding temperature value exceeds the thresholds set.

\subsubsection*{Average flow}
The average flow is tasked with the evaluation of the average temperature of each room and the house over one minute by taking the values from the global variable set in the previous flows.\\
The average of the house is immediately displayed on the dashboard and published to a Thingspeak channel while the average of each room are just published to the corresponding field in the Thingspeak channel. These values are retrieved by subscribing to the same Thingspeak channel and then it is displayed.\\
The average value are checked (the house average is checked immediately while the room average values are checked after retrieving them through MQTT) to see whether they are in the range set for the alarm and eventually a email is sent.

 

\section*{Node-RED import problems}
Node-RED does not allow us to export some node settings and for that reason some features may not completely work. 
To make our project work you have to: 
\begin{enumerate}
	\setcounter{enumi}{0}
	\item Set all email nodes user id to "smart.thermostat.iot.2019@gmail.com" and password to "IoT2019Af".
	\item Open the MQTT subscribe node and make sure that in the "thingspeak-subscribe" server settings the password is set to: P34OKQ312G55D98D.
\end{enumerate}


\end{document}
